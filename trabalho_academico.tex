% +-------------------------------------------------------------------------+
% | LICENÇA                                                                 |
% | abtex2-modelo-trabalho-academico.tex, v-1.9.2 laurocesar                |
% | Copyright 2012-2014 by abnTeX2 group at http://abntex2.googlecode.com/  |
% |                                                                         |
% | This work may be distributed and/or modified under the conditions of    |
% | the LaTeX Project Public License, either version 1.3 of this license or |
% | (at your option) any later version. The latest version of this license  |
% | is in http://www.latex-project.org/lppl.txt and version 1.3 or later    |
% | is part of all distributions of LaTeX version 2005/12/01 or later.      |
% |                                                                         |
% | This work has the LPPL maintenance status `maintained'.                 |
% |                                                                         |
% | The Current Maintainer of this work is the abnTeX2 team, led by Lauro   |
% | César Araujo. Further information are available on                      |
% | http://abntex2.googlecode.com/                                          |
% +-------------------------------------------------------------------------+
%
% +-------------------------------------------------------------------------+
% | INFORMAÇÕES                                                             |
% | Esse modelo de trabalho a acadêmico possui os seguintes arquivos:       |
% | PRINCIPAL (Diretório)                                                   |
% |    trabalho_academico.tex                                               |
% | ELEMENTOS PRÉ-TEXTUAIS (diretório)                                      |
% |    capa                        (obrigatório - via comando LaTeX)        |
% |    folha de rosto              (obrigatório - via comando LaTeX)        |
% |    ficha_catalografica.tex     (obrigatório - via biblioteca)           |
% |    errata.tex                  (opcional)                               |
% |    folha_de_aprovacao.tex      (obrigatório - documento externo)        |
% |    dedicatoria.tex             (opcional)                               |
% |    agradecimentos.tex          (opcional)                               |
% |    epigrafe.tex                (opcional)                               |
% |    resumo.tex                  (obrigatório)                            |
% |    abstract.tex                (obrigatório)                            |
% |    lista de figuras            (opcional - via comando LaTeX)           |
% |    lista de tabelas            (opcional - via comando LaTeX)           |
% |    abreviaturas.tex            (opcional)                               |
% |    simbolos.tex                (opcional)                               |
% |    sumário                     (obrigatório - via comando LaTeX)        |
% | ELEMENTOS TEXTUAIS (diretório)                                          |
% |    prefacio.tex                (opcional)                               |
% |    introducao.tex              (obrigatório)                            |
% |    desenvolvimento.tex         (obrigatório)                            |
% |    conclusao.tex               (obrigatório)                            |
% | ELEMENTOS PÓS-TEXTUAIS (diretório)                                      |
% |    apendice_a.tex              (opcional)                               |
% |    anexo_a.tex                 (opcional)                               |
% |    referencias bibliográficas  (obrigatório - via comando LaTeX)        |
% |    bibliografia.bib            (obrigatório)                            |
% |    glossário                   (opcional - via comando LaTeX)           |
% |    índice remissivo            (opcional - via comando LaTeX)           |
% +-------------------------------------------------------------------------+

\documentclass[%
% --- OPÇÕES DA CLASSE MEMOIR ---
12pt,      % tamanho da fonte
openright, % Capítulos começam em pág ímpar (insere página em branco em caso de necessidade)
oneside,	   % Impressão em verso e anverso [oneside]
a4paper,   % Tamanho do papel
% --- OPÇÕES DA CLASSE ABNTEX2 ---
% chapter=TITLE,       % Títulos de capítulos convertidos em letras maiúsculas
% section=TITLE,       % Títulos de seções convertidos em letras maiúsculas
% subsection=TITLE,    % Títulos de subseções convertidos em letras maiúsculas
% subsubsection=TITLE, % Títulos de subsubseções convertidos em letras maiúsculas
% --- OPÇÕES DO PACOTE BABEL ---
english,	 % Idioma adicional para hifenização
spanish,	 % Idioma adicional para hifenização
brazil	 % O último idioma é o principal do documento
]{abntex2}
%
%
%
% +---------+
% | Pacotes | 
% +---------+
% PACOTES BÁSICOS
\usepackage{lmodern} % Usa a fonte Latin Modern			
\usepackage[T1]{fontenc}	 % Selecao de codigos de fonte
\usepackage[utf8]{inputenc} % Codificacao do documento (conversão automática dos acentos)
\usepackage{lastpage} % Usado pela Ficha catalográfica
\usepackage{indentfirst}	 % Indenta o primeiro parágrafo de cada seção
\usepackage{color} % Controle das cores
\usepackage{graphicx} % Inclusão de figuras
\usepackage{microtype} % para melhorias de justificação
% PACOTES ADICIONAIS 
\usepackage{lipsum} % Para geração de dummy text
% PACOTES DE CITAÇÕES
\usepackage[brazilian,hyperpageref]{backref} % Páginas com as citações na bibl
\usepackage[alf]{abntex2cite} % Citações padrão ABNT

% +-------------------------------------+
% | CONFIGURAÇÕES DE PACOTES E COMANDOS |
% +-------------------------------------+
% --- CONFIGURAÇÕES DO PACOTE BACKREF ---
% Usado sem a opção hyperpageref de backref
\renewcommand{\backrefpagesname}{Citado na(s) página(s):~}
% Texto padrão antes do número das páginas
\renewcommand{\backref}{}
% Define os textos da citação
\renewcommand*{\backrefalt}[4]{
	\ifcase #1 %
		Nenhuma citação no texto.%
	\or
		Citado na página #2.%
	\else
		Citado #1 vezes nas páginas #2.%
	\fi}%

% +--------------------------------------------------------------+
% | INFORMAÇÕES GERAIS                                           |
% | Informações e dados para composição da CAPA e FOLHA DE ROSTO |
% +--------------------------------------------------------------+
\titulo{Modelo de trabalho acadêmico nas normas da ABNT}
\autor{Nome do(a) autor(a)}
\local{Brasil - Paraná - Paranaguá}
\data{31 de março de 2022, v-1.0}
\orientador{Nome do orientador}
\coorientador{Nome do coorientador}
\instituicao{%
  Instituto Federal do Paraná -- IFPR
  \par
  Curso de Licenciatura em Física}
\tipotrabalho{Trabalho de Conclusão de Curso}
\preambulo{ Trabalho de conclusão de curso apresentado ao Curso de Licenciatura em Física do Instituto Federal do Paraná/Paranaguá, como requisito parcial para a obtenção do grau de Licenciado(a) em Física.}

% --- CONFIGURAÇÕES DE APARÊNCIA DO PDF FINAL ---
% Alterando o aspecto da cor azul no PDF
\definecolor{blue}{RGB}{41,5,195}
%
% Informações do PDF
\makeatletter
\hypersetup{
%pagebackref=true,
pdftitle={\@title}, 
pdfauthor={\@author},
pdfsubject={\imprimirpreambulo},
pdfcreator={LaTeX with abnTeX2},
pdfkeywords={abnt}{latex}{abntex}{abntex2}{trabalho acadêmico}, 
colorlinks=true, % false: boxed links; true: colored links
linkcolor=blue, % Cor dos links internos
citecolor=blue, % Cor dos links das referências bibiiográficas
filecolor=blue, % Cor dos links para os arquivos
urlcolor=blue, % Cor dos links para páginas web
bookmarksdepth=4
}
\makeatother

% +---------------------+
% | FORMATAÇÃO DO TEXTO |
% +---------------------+
\setlength{\parindent}{1.3cm} % Tamanho do parágrafo
\setlength{\parskip}{0.2cm} % Controle do espaçamento entre um parágrafo e outro. Tente também \onelineskip

% +------------------+
% | ÍNDICE REMISSIVO |
% +------------------+
%\makeindex

% +---------------------+
% | INÍCIO DO DOCUMENTO |
% +---------------------+
\begin{document}
\frenchspacing % Controle do espaçamento entre um parágrafo e outro

% +------------------------+
% | ELEMENTOS PRÉ-TEXTUAIS |
% +------------------------+
\pretextual

\imprimircapa % (obrigatório)

\imprimirfolhaderosto* % (obrigatório) (o * indica que haverá a ficha bibliográfica)

% +---------------------------------------------------------------------------------+
% | FICHA CATALOGRÁFICA                                                             |
% | Apresentamos abaixo um exemplo de Ficha Catalográfica, ou "Dados internacionais |
% | de catalogação-na-publicação". Você pode utilizar este modelo como referência.  |
% | Porém, provavelmente a biblioteca da sua universidade lhe fornecerá um PDF      |
% | com a ficha catalográfica definitiva após a defesa do trabalho. Quando estiver  |
% | com o documento, salve-o como PDF no diretório do seu projeto e substitua todo  |
% | o conteúdo de implementação deste arquivo pelo comando abaixo.                  |
% +---------------------------------------------------------------------------------+
% \begin{fichacatalografica}
%     \includepdf{fig_ficha_catalografica.pdf}
% \end{fichacatalografica}
\begin{fichacatalografica}
	\vspace*{\fill}					% Posição vertical
	\hrule							% Linha horizontal
	\begin{center}					% Minipage Centralizado
	\begin{minipage}[c]{14.5cm}		% Largura
	
	\imprimirautor
	
	\hspace{0.5cm} \imprimirtitulo  / \imprimirautor. --
	\imprimirlocal, \imprimirdata-
	
	\hspace{0.5cm} \pageref{LastPage} p. : il. (algumas color.) ; 30 cm.\\
	
	\hspace{0.5cm} \imprimirorientadorRotulo~\imprimirorientador\\
	
	\hspace{0.5cm}
	\parbox[t]{\textwidth}{\imprimirtipotrabalho~--~\imprimirinstituicao,
	\imprimirdata.}\\
	
	\hspace{0.5cm}
		1. Palavra-chave1.
		2. Palavra-chave2.
		I. Orientador.
		II. Universidade xxx.
		III. Faculdade de xxx.
		IV. Título\\ 			
	
	\hspace{8.75cm} CDU 02:141:005.7\\
	
	\end{minipage}
	\end{center}
	\hrule
\end{fichacatalografica}
%
% EOF
% % (obrigatório - via biblioteca)

% +----------------------------------------------------------------------------------------------+
% | ERRATA                                                                                       |
% | A errata é um elemnto pré-textual opcional que deve ser inserido logo após a folha de rosto. |
% | Deve ser constituído por referência do trabalho e texto da errata.                           |
% +----------------------------------------------------------------------------------------------+
\begin{errata}
\vspace{\onelineskip}
%
%
%
% REFERÊNCIA DO TRABALHO
FERRIGNO, C. R. A. \textbf{Tratamento de neoplasias ósseas apendiculares com
reimplantação de enxerto ósseo autólogo autoclavado associado ao plasma
rico em plaquetas}: estudo crítico na cirurgia de preservação de membro em
cães. 2011. 128 f. Tese (Livre-Docência) - Faculdade de Medicina Veterinária e
Zootecnia, Universidade de São Paulo, São Paulo, 2011.
%
%
%
% TEXTO DA ERRATA
\begin{table}[htb]
\center
\footnotesize
\begin{tabular}{|p{1.4cm}|p{1cm}|p{3cm}|p{3cm}|}
  \hline
   \textbf{Folha} & \textbf{Linha}  & \textbf{Onde se lê}  & \textbf{Leia-se}  \\
    \hline
    1 & 10 & auto-conclavo & autoconclavo\\
   \hline
\end{tabular}
\end{table}

\end{errata}
%
% EOF
% % (opcional)

% +----------------------------------------------------------------------------+
% | FOLHA DE APROVAÇÃO                                                         |
% | Este é um exemplo de folha de aprovação, elemento obrigatório da NBR       |
% | 14724/2011 (seção 4.2.1.3). Você pode utilizar este modelo até a aprovação |
% | do trabalho. Após isso, substitua todo o conteúdo deste arquivo por uma    |
% | imagem da página assinada pela banca com o comando abaixo.                 |
% +----------------------------------------------------------------------------+
% \includepdf{folhadeaprovacao_final.pdf}
%
\begin{folhadeaprovacao}

  \begin{center}
    {\ABNTEXchapterfont\large\imprimirautor}

    \vspace*{\fill}\vspace*{\fill}
    \begin{center}
      \ABNTEXchapterfont\bfseries\Large\imprimirtitulo
    \end{center}
    \vspace*{\fill}
    
    \hspace{.45\textwidth}
    \begin{minipage}{.5\textwidth}
        \imprimirpreambulo
    \end{minipage}%
    \vspace*{\fill}
   \end{center}
        
   Trabalho aprovado. \imprimirlocal:

   \assinatura{\textbf{\imprimirorientador} \\ Orientador} 
   \assinatura{\textbf{Professor} \\ Convidado 1}
   \assinatura{\textbf{Professor} \\ Convidado 2}
   %\assinatura{\textbf{Professor} \\ Convidado 3}
   %\assinatura{\textbf{Professor} \\ Convidado 4}
      
   \begin{center}
    \vspace*{0.5cm}
    {\large\imprimirlocal}
    \par
    {\large\imprimirdata}
    \vspace*{1cm}
  \end{center}
  
\end{folhadeaprovacao}
%
% EOF
% % (obrigatório - documento externo)

% +-------------+
% | DEDICATÓRIA |
% +-------------+
\begin{dedicatoria}
\vspace*{\fill}
\centering
\noindent
\textit{ Este trabalho é dedicado às crianças adultas que,\\
quando pequenas, sonharam em se tornar cientistas.}
\vspace*{\fill}
\end{dedicatoria}
%
% EOF
% % (opcional)

% +----------------+
% | AGRADECIMENTOS |
% +----------------+
\begin{agradecimentos}
\lipsum[1-3]
\end{agradecimentos}
%
% EOF
% % (opcional)

% +----------+
% | EPÍGRAFE |
% +----------+
\begin{epigrafe}
\vspace*{\fill}
	\begin{flushright}
	\textit{Todos sabemos que a arte não contém verdades.\\
	A arte é uma mentira que nos ajuda a ver a verdade,\\
	pelo menos a verdade que nos é dado compreender.\\
	O artista tem de saber como convencer\\
	os outros da veracidade das suas mentiras.\\
	\vspace*{3mm}	
	Pablo Picasso}
	\end{flushright}
\end{epigrafe}
%
% EOF
% % (opcional)

% +---------+
% | RESUMOS |
% +---------+
% O resumo deverá ser apresentado na língua vernáculo (Português).
\setlength{\absparsep}{18pt} % Ajusta o espaçamento dos parágrafos do resumo
\begin{resumo}
\lipsum[1-2]

\textbf{Palavras-chaves}: Física. abntex. editoração de texto.
\end{resumo}
%
% EOF
% % (obrigatório)

% +----------+
% | ABSTRACT |
% +----------+
\begin{resumo}[Abstract]
 \begin{otherlanguage*}{english}
   \lipsum[1-2]
 
   \noindent 
   \textbf{Keywords}: latex. abntex. text editoration.
 \end{otherlanguage*}
\end{resumo}
%
% EOF
% % (obrigatório)

% --- LISTA DE FIGURAS ---
%\pdfbookmark[0]{\listfigurename}{lof}
%\listoffigures* % (opcional)
%\cleardoublepage

% --- LISTA DE TABELAS ---
%\pdfbookmark[0]{\listtablename}{lot}
%\listoftables* % (opcional)
%\cleardoublepage

% +--------------------------------+
% | LISTA DE SIGLAS E ABREVIATURAS |
% +--------------------------------+
\begin{siglas}
  \item[ABNT] Associação Brasileira de Normas Técnicas
  \item[SBF] Sociedade Brasileira de Física
\end{siglas}
%
% EOF
% % (opcional)

% +-------------------+
% | LISTA DE SÍMBOLOS |
% +-------------------+
\begin{simbolos}
  \item[$ \Gamma $] Letra grega Gama
  \item[$ \Lambda $] Lambda
  \item[$ \zeta $] Letra grega minúscula zeta
  \item[$ \in $] Pertence
\end{simbolos}
%
% EOF
% % (opcional)

% --- SUMÁRIO ---
\pdfbookmark[0]{\contentsname}{toc}
\tableofcontents* % (obrigatório)
\cleardoublepage

% +--------------------+
% | ELEMENTOS TEXTUAIS |
% +--------------------+
\textual
\pagestyle{simple}
% +----------+
% | PREFÁCIO |
% +----------+
\chapter*[Prefácio]{Prefácio}
\addcontentsline{toc}{chapter}{Prefácio}

\lipsum
Consulte as Ref.~\cite{gil_2010,oliveira_2011,calcada_2005}.
%
% EOF
% % (opcional)

% +------------+
% | INTRODUÇÃO |
% +------------+
\chapter{Introdução}
\lipsum

\section{Inclusão da seção}
\lipsum[1-6]

\section{Inclusão da seção}
\lipsum[1-6]
%
% EOF
% % (obrigatório)

% +-----------------+
% | DESENVOLVIMENTO |
% +-----------------+
\chapter{Desenvolvimento}
\lipsum[1-3]

\section{Seção de desenvolvimento}
\lipsum

\section{Seção de desenvolvimento}
\lipsum
%
% EOF
% % (obrigatório)

% +-----------+
% | CONCLUSÃO |
% +-----------+
\chapter*[Conclusão]{Conclusão}
\addcontentsline{toc}{chapter}{Conclusão}

\lipsum

\lipsum
%
% EOF
% % (obrigatório)

% +------------------------+
% | ELEMENTOS PÓS-TEXTUAIS |
% +------------------------+
\postextual

% +-----------+
% | APÊNDICES |
% +-----------+
\begin{apendicesenv}
\partapendices % Imprime uma página indicando o início dos apêndices
% +------------+
% | APÊNDICE A |
% +------------+
\chapter{Título do apêndice}
\lipsum

%
% EOF
%
\end{apendicesenv}

% +--------+
% | ANEXOS |
% +--------+
\begin{anexosenv}
\partanexos % Imprime uma página indicando o início dos anexos
% +---------+
% | ANEXO A |
% +---------+
\chapter{Título do anexo}
\lipsum[1-8]
%
% EOF
%
\end{anexosenv}

% +----------------------------+
% | REFERÊNCIAS BIBLIOGRÁFICAS |
% +----------------------------+
\bibliography{pos_textuais/bibliografia}

% +-----------+
% | GLOSSÁRIO |
% +-----------+
% Consulte o manual da classe abntex2 para orientações sobre o glossário.
%\phantompart
%\glossary % (opcional)

% +------------------+
% | ÍNDICE REMISSIVO |
% +------------------+
% Consulte o manual do LaTeX para orientações sobre elaboração do índice remissivo.
%\phantompart
%\printindex % (opcional)
\end{document}
%
% EOF
%